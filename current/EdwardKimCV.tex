%%%%%%%%%%%%%%%%%%%%%%%%%%%%%%%%%%%%%%%%%
% Medium Length Professional CV
% LaTeX Template
% Version 2.0 (8/5/13)
%
% This template has been downloaded from:
% http://www.LaTeXTemplates.com
%
% Original author:
% Trey Hunner (http://www.treyhunner.com/)
%
% Important note:
% This template requires the resume.cls file to be in the same directory as the
% .tex file. The resume.cls file provides the resume style used for structuring the
% document.
%
%%%%%%%%%%%%%%%%%%%%%%%%%%%%%%%%%%%%%%%%%

%----------------------------------------------------------------------------------------
%	PACKAGES AND OTHER DOCUMENT CONFIGURATIONS
%----------------------------------------------------------------------------------------

\documentclass{resume} % Use the custom resume.cls style

\usepackage[left=0.4 in,top=0.4in,right=0.4 in,bottom=0.4in]{geometry} % Document margins
\usepackage{hyperref}
\newcommand{\tab}[1]{\hspace{.2667\textwidth}\rlap{#1}}
\newcommand{\itab}[1]{\hspace{0em}\rlap{#1}}
\name{Edward Kim} % Your name
\address{{\bf Address:} Dept. of Computer Science. UNC-Chapel Hill, Chapel Hill, NC 27599-3175}
\address{{\bf Email:} ehkim@cs.unc.edu}
\address{\href{https://ekim1919.github.io}{Webpage}, \href{https://scholar.google.com/citations?user=Wn6iETgAAAAJ&hl=en&authuser=2}{Google Scholar}}

\begin{document}

%----------------------------------------------------------------------------------------
%	EDUCATION SECTION
%----------------------------------------------------------------------------------------

\begin{rSection}{Education}

{\bf The University of North Carolina at Chapel Hill} \hfill{\bf 2019 -} \\
Graduate Student, Computer Science

{\bf The University of California at Berkeley} \hfill {\bf 2013 - 2017} \\
B.A in Computer Science \\
Honors B.A in Pure Mathematics \\
{\small Specialized in Theoretical CS, Mathematical Logic}
\end{rSection}

\begin{rSection}{Research Interests}

\begin{tabular}{ @{} >{\bfseries}l @{\hspace{6ex}} l }
{\bf Quantum Computation Theory:} & Topological Quantum Computation \\
                                  & Quantum Error-Correction\\
                                  & Quantum Algorithms \\
                                  & Quantum Algebra and Topology \\
\end{tabular}

\end{rSection}

\begin{rSection}{Publications}
  \begin{itemize}
    \item {\bf Edward Kim}, Stanley Bak, Parasara Sridhar Duggirala: Automatic Dynamic Parallelotope Bundles for Reachability Analysis of Nonlinear Systems, \emph{In submission}, arXiv preprint arXiv:2105.11796
    \item {\bf Edward Kim}, Parasara Sridhar Duggirala: Kaa: A Python Implementation of Reachable Set Computation Using Bernstein Polynomials, 7th Int. Workshop on Applied Verification for Continuous and Hybrid Systems, 2020
    \item Wei-Kai Lai, {\bf Edward Kim}:  Some inequalities involving geometric and harmonic means, International Mathematical Forum, Vol. 11, 2016, no. 4, 163-169
  \end{itemize}
\end{rSection}

\begin{rSection}{Other Contributions}
  \begin{itemize}
    \item Stanley Bak, {\bf Edward Kim}, Parasara Sridhar Duggirala: COVID Infection Prediction using CPS Formal Verification Methods, ACM SIGBED Blog, June 21, 2021. \href{https://sigbed.org/2021/06/21/sidbed-blog-covid-formal-verification/}{Link}
  \end{itemize}
\end{rSection}

\begin{rSection}{Expositions}
  \begin{itemize}
    \item {\bf Notes on the Fourier Analysis of Boolean Functions}
    \hfill \\
    Wrote an short survey on the Fourier Analysis of Boolean Functions with view towards the Linial-Mansour-Nisan Theorem.
    \href{https://github.com/ekim1919/Research/blob/master/CS590/EdwardKimPaper.pdf}{Link to Report}
    \item {\bf Quantum Expanders and $k$-designs} \hfill \\
    Wrote an expository paper on the theory of Quantum Pseudorandomess as a project for Advanced Topics in Quantum Information.
    \href{https://github.com/ekim1919/Research/blob/master/PHYS790/final.pdf}{Link to Report}
    %\item{\bf  Schur-Weyl Duality in Quantum Information} \hfill \\
    %Wrote an expository paper on applications of the Schur-Weyl Duality to fundamental questions in %Quantum Information Theory. Presented it to the Duke University PHYS 590 class of Spring 2020.
    %\href{https://github.com/ekim1919/Research/blob/master/P590/final.pdf}{Link to Report}
  \end{itemize}
%{\bf Personal Lecture Notes} \hfill \\
%Created extensive lecture notes for personal edification. Covers a wide variety of topics such as quantum computing, recursion theory, algebraic topology. \\ \href{https://github.com/ekim1919/Notes}{Link to Notes} \
\end{rSection}

\begin{rSection}{Relevant Course Work}

\begin{tabular}{ @{} >{\bfseries}l @{\hspace{6ex}} l }
Mathematics:
& Lie Groups, Smooth Manifolds, Measure Theory, Functional Analysis,  \\
& Differential Geometry, Lie Algebras and their Representations, \\
& Homological Algebra, Commutative Algebra, Complex Analysis \\
\\
Computer Science: & Quantum Algorithms and Computation, Quantum Information Theory, \\
& Machine Learning, Boolean Function Complexity, \\
& Computational Complexity Theory, Randomized Algorithms  \\
\end{tabular}
\end{rSection}
\newpage


\begin{rSection}{Experience}
{\bf University of North Carolina at Chapel Hill} \hfill {\bf 2019 -} \\
{\bf Research Assistant}- Providing research assistance to projects pertaining to the formal verification of safety properties of non-linear cyber-physical systems. Focusing on counter-example generation to aid practitioners in verifying the safety of their models.
\begin{itemize}
\item Created a tool called Kaa for the reachability of non-linear discrete dynamical systems using parallelotope bundles. Improved on existing tools for reachablility computation using these techniques.
\item Contributed to the documentation efforts of HyLAA, a verification tool of hybrid automata governed by linear dynamics.
\end{itemize}

{\bf University of South Carolina} \hfill {\bf 2016} \\
{\bf Research Assistant}- Published some basic results about fundamental inequalities by remotely collaborating with Professor Wei-Kai Lai from the University of South Carolina, Salkehatchie. \\
\end{rSection}

\begin{rSection}{Services}
  {\bf UNC Cyber-Physical Systems Lunch Organizer} \hfill {\bf Spring 2021}\\
  Organized the Cyber-physical/Real-time systems lunch where students in Autonomous Systems, Real-Time systems, Cyber-physical systems, and Formal Verifiction met to discuss research and present recently-published papers during an hour-long session.
\end{rSection}

\begin{rSection}{Pedagogy}
{\bf Calculus Tutor} \hfill {\bf 2018} \\
Tutored Calculus to students at South Carolina State University. Stressed geometric intuition and visual approaches rather than rote memorization of formulae and concepts. \\
\\
{\bf Programming Languages Tutor} \hfill {\bf 2018} \\
Provided discussions for South Carolina State University Computer Science students attending summer courses. Discussions pertained to Python, Java, and C. \\
\end{rSection}


\begin{rSection}{Programming Languages and Skills [\href{https://github.com/ekim1919}{Github}]}
  \begin{description}
    \item[Proficient Languages] - C/C\texttt{++}, C\#, Python, Haskell, Java.
    \item[\bf Kaa ($>$ 5000 lines, Python)] - Software created to experiment with reachable set computations of non-linear systems governed under discrete polynomial dynamics. The project was specifically created to understand the effectiveness of dynamically-reorienting parallelotope bundles on improving the quality over-approximations of reachable sets. It is the first experimental software created to properly plot the evolution of these dynamic bundle strategies for practitioners to understand the efficacy of different bundle strategies. It significantly improved the usability of previously existing reachable set simulators using these techniques.
    \item[TDAGo (Python)] - Python program to analyze Go games using Persistence Homology for Duke's Topological Data Analysis class. Used the evolution of persistence diagrams to detect topologically-significant features of games played between iterations of Google Deepmind's AlphaGo.\href{https://github.com/ekim1919/TDAGo/blob/master/paper/final.pdf}{ [Link to Report] }
    %\item[Simple Microarchitecture (Assembly)] - Programmed microcode implementing the ISA for a simple microarchitecture completely from %stratch. Includes a program simluating the CPU by clock-cycle to visualize the microcode working.
  \end{description}
\end{rSection}

%\begin{rSection}{Workshops/Conferences Attended}
%{\bf Simons Institute for the Theory of Computation} \hfill {\bf 2020} \\
%Spring 2020 Workshop on Quantum Protocols: Testing \& Quantum PCPs \\
%\href{https://simons.berkeley.edu/workshops/quantum-2020-2}{Link to Workshop Description}
%\end{rSection}

\end{document}
