%%%%%%%%%%%%%%%%%%%%%%%%%%%%%%%%%%%%%%%%%
% Medium Length Professional CV
% LaTeX Template
% Version 2.0 (8/5/13)
%
% This template has been downloaded from:
% http://www.LaTeXTemplates.com
%
% Original author:
% Trey Hunner (http://www.treyhunner.com/)
%
% Important note:
% This template requires the resume.cls file to be in the same directory as the
% .tex file. The resume.cls file provides the resume style used for structuring the
% document.
%
%%%%%%%%%%%%%%%%%%%%%%%%%%%%%%%%%%%%%%%%%

%----------------------------------------------------------------------------------------
%	PACKAGES AND OTHER DOCUMENT CONFIGURATIONS
%----------------------------------------------------------------------------------------

\documentclass{resume} % Use the custom resume.cls style

\usepackage[left=0.4 in,top=0.4in,right=0.4 in,bottom=0.4in]{geometry} % Document margins
\usepackage{hyperref}
\newcommand{\tab}[1]{\hspace{.2667\textwidth}\rlap{#1}}
\newcommand{\itab}[1]{\hspace{0em}\rlap{#1}}
\name{Edward Kim} % Your name
\address{{\bf Address:} Dept. of Computer Science. UNC-Chapel Hill, Chapel Hill, NC 27599-3175}
\address{{\bf Email:} ehkim@cs.unc.edu}
%\address{Webpage: cs.unc.edu/ehkim}

\begin{document}

%----------------------------------------------------------------------------------------
%	EDUCATION SECTION
%----------------------------------------------------------------------------------------

\begin{rSection}{Education}

{\bf The University of North Carolina at Chapel Hill} \hfill{\bf 2019 -} \\
Ph.D Candidate, Computer Science

{\bf The University of California at Berkeley} \hfill {\bf 2013 - 2017} \\
B.A in Computer Science \\
Honors B.A in Pure Mathematics \\
{\small Specialized in Theoretical CS, Mathematical Logic}
\end{rSection}

\begin{rSection}{Research Interests}

\begin{tabular}{ @{} >{\bfseries}l @{\hspace{6ex}} l }
{\bf Quantum Computation Theory:} & Quantum Error Correction \\ & Quantum Information Theory \\
& Topological Quantum Computation \\
\\
{\bf Computational Complexity Theory: } & Quantum Complexity Theory \\ & Geometric Complexity Theory
\end{tabular}

\end{rSection}

\begin{rSection}{Publications}
  \begin{itemize}
    \item {\bf Edward Kim}, Parasara Sridhar Duggirala: Kaa: A Python Implementation of Reachable Set Computation Using Bernstein Polynomials, 7th Int. Workshop on Applied Verification for Continuous and Hybrid Systems, 2020
  \end{itemize}
\end{rSection}

\begin{rSection}{Expositions}
  \begin{itemize}
    \item {\bf Notes on the Fourier Analysis of Boolean Functions}
    \hfill \\
    Wrote an short survey on the Fourier Analysis of Boolean Functions with view towards the Linial-Mansour-Nisan Theorem.
    \href{https://github.com/ekim1919/Research/blob/master/CS590/EdwardKimPaper.pdf}{Link to Report}
    \item {\bf Quantum Expanders and $k$-designs} \hfill \\
    Wrote an expository paper on the theory of Quantum Pseudorandomess as a project for Advanced Topics in Quantum Information.
    \href{https://github.com/ekim1919/Research/blob/master/PHYS790/final.pdf}{Link to Report}
    \item{\bf  Schur-Weyl Duality in Quantum Information} \hfill \\
    Wrote an expository paper on applications of the Schur-Weyl Duality to fundamental questions in Quantum Information Theory. Presented it to the Duke University PHYS 590 class of Spring 2020.
\href{https://github.com/ekim1919/Research/blob/master/P590/final.pdf}{Link to Report}
  \end{itemize}
%{\bf Personal Lecture Notes} \hfill \\
%Created extensive lecture notes for personal edification. Covers a wide variety of topics such as quantum computing, recursion theory, algebraic topology. \\ \href{https://github.com/ekim1919/Notes}{Link to Notes} \
\end{rSection}

\begin{rSection}{Relevant Course Work}

\begin{tabular}{ @{} >{\bfseries}l @{\hspace{6ex}} l }
Mathematics:
& Recursion Theory, Model Theory, Lie Groups, Smooth Manifolds,  \\
& Measure Theory, Functional Analysis, Differential Geometry   \\
& Lie Algebras and their Representations, Homological Algebra, \\
& Algebraic Topology, Mathematical Logic, Complex Analysis\\
\\
Computer Science: & Quantum Algorithms and Computation, Computational Complexity Theory, \\
& Automata Theory and Computability, Algorithms in Computational Biology, \\
& Topological Data Analysis, Boolean Function Complexity, \\
& Quantum Information Theory, Randomized Algorithms  \\
\end{tabular}
\end{rSection}
\newpage
\begin{rSection}{Software}
  \begin{description}
    \item[Proficient Languages] - C, C\#, Python, Haskell, Common Lisp, Java
    \item[\bf Kaa ($>$ 5000 lines, Python)] - Software created to experiment with reachable set computations of non-linear systems governed under discrete polynomial dynamics. It was specifically created to understand the effectiveness of dynamically-reorienting parallelotope bundles on improving the quality of reachable set over-approximations. It is the first experimental software created to properly plot the evolution of these dynamic bundle strategies for practitioners to understand the efficacy of different bundle strategies. It significantly improved the usability of previously existing reachable set simulators using these parallelotope bundles. \\
  \end{description}
\end{rSection}

\begin{rSection}{Work Experience}
{\bf University of North Carolina at Chapel Hill} \hfill {\bf 2019 -} \\
{\bf Research Assistant}- Providing research assistance to projects pertaining to the formal verification of safety properties of non-linear cyber-physical systems. Focusing on counter-example generation to aid practitioners in verifying the safety of their models. Supervised by Parasara Sridhar Duggirala.
\begin{itemize}
\item Created a tool called Kaa for the reachability of non-linear discrete dynamical systems using parallelotope bundles.
\item Contributed to the documentation efforts of HyLAA, a verification tool of hybrid automata governed by linear dynamics.
\end{itemize}

{\bf University of South Carolina} \hfill {\bf 2016} \\
{\bf Research Assistant}- Published some basic results about fundamental inequalities by remotely collaborating with Professor Wei-Kai Lai from the University of South Carolina, Salkehatchie. \\
Paper: \href{http://www.m-hikari.com/imf/imf-2016/1-4-2016/51190.html}{Some inequalities involving geometric and harmonic means}
\end{rSection}

\begin{rSection}{Pedagogy}
{\bf Calculus Tutor} \hfill {\bf 2018} \\
Tutored Calculus to students at South Carolina State University. Stressed geometric intuition and visual approaches rather than rote memorization of formulae and concepts. \\
\\
{\bf Programming Languages Tutor} \hfill {\bf 2018} \\
Provided discussions for South Carolina State University Computer Science students attending summer courses. Discussions pertained to Python, Java, and C. \\
\end{rSection}


\begin{rSection}{Workshops/Conferences Attended}
{\bf Simons Institute for the Theory of Computation} \hfill {\bf 2020} \\
Spring 2020 Workshop on Quantum Protocols: Testing \& Quantum PCPs \\
\href{https://simons.berkeley.edu/workshops/quantum-2020-2}{Link to Workshop Description}
\end{rSection}

\end{document}
