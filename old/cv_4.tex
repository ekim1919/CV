%%%%%%%%%%%%%%%%%%%%%%%%%%%%%%%%%%%%%%%%%
% Medium Length Professional CV
% LaTeX Template
% Version 2.0 (8/5/13)
%
% This template has been downloaded from:
% http://www.LaTeXTemplates.com
%
% Original author:
% Trey Hunner (http://www.treyhunner.com/)
%
% Important note:
% This template requires the resume.cls file to be in the same directory as the
% .tex file. The resume.cls file provides the resume style used for structuring the
% document.
%
%%%%%%%%%%%%%%%%%%%%%%%%%%%%%%%%%%%%%%%%%

%----------------------------------------------------------------------------------------
%	PACKAGES AND OTHER DOCUMENT CONFIGURATIONS
%----------------------------------------------------------------------------------------

\documentclass{resume} % Use the custom resume.cls style

\usepackage[left=0.4 in,top=0.4in,right=0.4 in,bottom=0.4in]{geometry} % Document margins
\usepackage{hyperref}
\newcommand{\tab}[1]{\hspace{.2667\textwidth}\rlap{#1}}
\newcommand{\itab}[1]{\hspace{0em}\rlap{#1}}
\name{Edward Kim} % Your name
\address{201 S. Columbia St. UNC-Chapel Hill, Chapel Hill, NC 27599-3175}
\address{Email: ehkim@cs.unc.edu}
%\address{Webpage: cs.unc.edu/ehkim}

\begin{document}

%----------------------------------------------------------------------------------------
%	EDUCATION SECTION
%----------------------------------------------------------------------------------------

\begin{rSection}{Education}

{\bf University of North Carolina at Chapel Hill} \hfill{\bf 2019 -} \\
Ph.D Candidate

{\bf University of California at Berkeley} \hfill {\bf 2013 - 2017} \\
B.A in Computer Science \\
Honors B.A in Pure Mathematics
\end{rSection}


\begin{rSection}{Relevant Course Work}

\begin{tabular}{ @{} >{\bfseries}l @{\hspace{6ex}} l }
Mathematics: & Recursion Theory, Model Theory, Graduate Algebra, Introduction to Smooth Manifolds,  \\
& Graduate Real Analysis, Functional and Fourier Analysis, Topological Data Analysis, \\
& Lie Algebras and their Representations, A First Course in Homological Algebra, \\
& Elementary Algebraic Geometry, Mathematical Logic, Numerical Analysis, \\
& Complex Analysis \\
\\
Computer Science: & Quantum Algorithms and Computation, Computational Complexity Theory, \\
& Theory of Computation, Algorithms in Computational Biology, \\
& Structural Complexity Theory, Boolean Function Complexity  \\
\end{tabular}

\end{rSection}


\begin{rSection}{Research Interests}

\begin{tabular}{ @{} >{\bfseries}l @{\hspace{6ex}} l }
{\bf Quantum Computation Theory:} & Quantum Complexity Theory, Quantum Information Theory, \\
& Theoretical Quantum Computation Models \\
\\
{\bf Computational Complexity Theory} & Geometric Complexity Theory, \\
& Interactive Protocols and Probablisitically-checkable Proofs   \\
\end{tabular}

\end{rSection}

\begin{rSection}{Research Experience}

{\bf University of North Carolina at Chapel Hill} \hfill {\bf 2019 -} \\
{\bf Research Assistant}- Providing research assistance to projects pertaining to the formal verification of nonlinear hybrid systems. Contributing to the development of HyLAA, a formal verification tool computing simulation-equivalent reachable sets for linear dynamical systems. Supervised by Parasara Sridhar Duggirala. \\
\\
{\bf University of South Carolina} \hfill {\bf 2016} \\
{\bf Research Assistant}- Garnered experience with proactive remote communication to publish some basic results concerning fundamental inequalities. This paper is joint-work with Professor Wei-Kai Lai.\\
Paper: \href{http://www.m-hikari.com/imf/imf-2016/1-4-2016/51190.html}{Some inequalities involving geometric and harmonic means}
\end{rSection}


\begin{rSection}{Teaching and Volunteer Experience}
{\bf Calculus Tutor} \hfill {\bf 2018} \\
Tutored Calculus to students at South Carolina State University. Stressed geometric intuition and visual approaches rather than rote memorization of formulae and concepts. \\
\\
{\bf Programming Languages Tutor} \hfill {\bf 2018} \\
Provided discussions for South Carolina State University Computer Science students attending summer courses. Discussions pertained to Python, Java, and C. \\
\end{rSection}
\newpage
\begin{rSection}{Extracurricular Activities}
{\bf Personal Lecture Notes} \hfill \\
Created extensive lecture Notes for personal edification. Covers wide va- riety of topics from algebraic topology to recursion theory. \\ \url{https://github.com/ekim1919/Notes} \\
\\
{\bf Expository Notes and Presentation on Hopkins-Levitzski Theorem} \hfill \\
Wrote an expository paper on the Hopkins-Levitzski Theorem as a primer on Artinian Rings. Presented it to the Elementary Algebraic Geometry class (Math 143). \\
\url{https://github.com/KitToast/Research/blob/master/143/paper.pdf}
\end{rSection}

\end{document}
